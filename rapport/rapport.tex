\documentclass[12pt]{article}
\usepackage[utf8]{inputenc}
\usepackage[french]{babel}
\usepackage[a4paper,left=2cm,right=2cm,top=2cm,bottom=2cm]{geometry}
\usepackage{indentfirst}
\usepackage{appendix}
\usepackage{libertine}
\usepackage{titlesec}
\usepackage{eso-pic}
\usepackage{fancyhdr}
\usepackage{graphicx}
\usepackage{wrapfig}
\usepackage{float}
\usepackage{caption}
\usepackage{color}
\usepackage{setspace}
\usepackage[T1]{fontenc}
\usepackage[hyphens]{url}
\usepackage[backend=bibtex,style=alphabetic]{biblatex}
\addbibresource{source.bib}\usepackage[linkbordercolor=white]{hyperref}
\usepackage{hyperref}
\usepackage{caption}
\usepackage{subcaption}
\usepackage{bm}
\usepackage{enumitem}
\usepackage{siunitx}
\usepackage{textcomp}
\usepackage{amsmath}
\usepackage{amssymb}
\usepackage{amsfonts}
\usepackage{amsthm}
\usepackage{commath}
\usepackage{cancel}
\usepackage{pdflscape}
\usepackage{svg}
\usepackage[linesnumbered,ruled,vlined]{algorithm2e} % Retain only one package
\usepackage{comment}
\usepackage{placeins}
\usepackage{listings}
\lstset{literate=%
    {à}{{\`a}}1 {â}{{\^a}}1 {é}{{\'e}}1 {è}{{\`e}}1 {ê}{{\^e}}1 {ë}{{\"e}}1 {î}{{\^i}}1 {ï}{{\"i}}1 {ô}{{\^o}}1 {ö}{{\"o}}1 {ù}{{\`u}}1 {û}{{\^u}}1 {ü}{{\"u}}1 {ç}{{\c{c}}}1 {Ç}{{\c{C}}}1{―}{{\textemdash}}1
    {’}{{'}}1 {‘}{{`}}1 {“}{{``}}1 {”}{{''}}1
}
\urlstyle{same}
\pagestyle{fancy}
\fancyhead[L]{\leftmark}
\fancyhead[R]{}
\renewcommand{\subsectionmark}[1]{}


\usepackage{titlesec}
\usepackage{hyperref}

\titleclass{\subsubsubsection}{straight}[\subsection]


\newcounter{subsubsubsection}[subsubsection]
\renewcommand\thesubsubsubsection{\thesubsubsection.\arabic{subsubsubsection}}
\renewcommand\theparagraph{\thesubsubsubsection.\arabic{paragraph}} % optional; useful if paragraphs are to be numbered


\titleformat{\subsubsubsection}
  {\normalfont\normalsize\bfseries}{\thesubsubsubsection}{1em}{}
\titlespacing*{\subsubsubsection}
{0pt}{3.25ex plus 1ex minus .2ex}{1.5ex plus .2ex}
\makeatletter
\renewcommand\paragraph{\@startsection{paragraph}{5}{\z@}%
  {3.25ex \@plus1ex \@minus.2ex}%
  {-1em}%
  {\normalfont\normalsize\bfseries}}
\renewcommand\subparagraph{\@startsection{subparagraph}{6}{\parindent}%
  {3.25ex \@plus1ex \@minus .2ex}%
  {-1em}%
  {\normalfont\normalsize\bfseries}}
\def\toclevel@subsubsubsection{4}
\def\toclevel@paragraph{5}
\def\toclevel@paragraph{6}
\def\l@subsubsubsection{\@dottedtocline{4}{7em}{4em}}
\def\l@paragraph{\@dottedtocline{5}{10em}{5em}}
\def\l@subparagraph{\@dottedtocline{6}{14em}{6em}}
\makeatother
\setcounter{secnumdepth}{4}
\setcounter{tocdepth}{4}

\setlength{\headheight}{14.5pt}
\setlength{\parskip}{1em}
\renewcommand{\baselinestretch}{0.6}
\renewcommand{\headrulewidth}{1pt}
\newcommand{\HRule}{\rule{\linewidth}{0.3mm}}
\newcommand\BackgroundPic{%
        \put(0,0){%
                \parbox[b][\paperheight]{\paperwidth}{%
                        \vfill
                        \centering
                        \includegraphics[width=\paperwidth,height=\paperheight,%
                        keepaspectratio]{background.png}%
                        \vfill}}}

% -----------------------------------------------------

\begin{document}
\AddToShipoutPicture*{\BackgroundPic}
\begin{titlepage}
  \begin{sffamily}
  \begin{flushleft} \large
    \includegraphics[height=2.0cm]{logo_ulb.jpg}
    \vspace{5cm}
   \end{flushleft}
  \begin{center}

    %Title
        \textsc{\huge INFO-F308 - Projet d'année}\\[1cm]

    \HRule \\[0.7cm]

        \textsc {\Huge Simulation de la réalité}\\[0.4cm]

    \HRule \\[1.2cm]

% Author and supervisor
\begin{minipage}{0.5\textwidth}
\begin{flushleft} \large
\emph{Auteurs:}\\
Romain \textsc{Liefferinckx} - 000591790\\
Manuel \textsc{Rocca} - 000\\
Matteo \textsc{Morbée} - 000\\
Denise \textsc{} - 000\\
Martin \textsc{Gouverneur} - 000\\


\end{flushleft}
\end{minipage}


\begin{minipage}{0.4\textwidth}
\begin{flushright} \large
\emph{Professeurs:} \\
Gilles  \textsc{Geraerts}\\
\emph{Promoteur:} \\
Pascal \textsc{Tribel}
\end{flushright}
\end{minipage}


    \vfill

    %Bottom of the page
    {\large Année académique 2025-2026}
  \end{center}

  \end{sffamily}
\end{titlepage}


\clearpage


\tableofcontents
\newpage

% -----------------------------------------------------
\section{Introduction}
Dans le cadre de notre cours de projet d'année, nous avons choisi comme sujet la simulation de la réalité. 
Nous en sommes ainsi venus à développer un simulateur physique capable de représenter\dots

\section{Equations utilisées}
Dans cette section, nous détaillons les équations qui nous ont permises de modéliser les différents phénomènes physiques dans notre simulateur.

fonction représentant notre drap:
\[
f(x, y) = A \exp\left(-\frac{(x - x_0)^2 + (y - y_0)^2}{2\sigma^2}\right)
\]

\begin{itemize}
    \item $A$ : amplitude = $-depth$ (négative puisqu'on veut un puit)
    \item $(x_0, y_0)$ : centre de la gaussienne = $(0, 0)$
    \item $\sigma$ : écart-type = ``largeur'' de la cloche
\end{itemize}

après simplification, nous arrivons à la fonction qui modélise notre puit:
\[
f(x, y) = -depth \exp\left(-\frac{x^2 + y^2}{2\sigma^2}\right)
\]

Passons maintenant aux équations de la physique newtonienne que nous avons utilisées dans notre simulateur.

En deux dimensions, on utilise la deuxième loi de Newton pour modéliser le mouvement des objets comme ceci:
\[
F = ma
\]
où
\begin{align*}
F &= \text{la force} \\
m &= \text{la masse de l'objet} \\
a &= \text{l'accélération de l'objet}
\end{align*}


\[
F_{grav} = -G \frac{M m}{r^2}
\]
où
\begin{align*}
F_{grav} &= \text{la force gravitationnelle} \\
G &= \text{la constante gravitationnelle} \\
M &= \text{la masse de la source de gravité} \\
m &= \text{la masse de l'objet attiré par la source} \\
r &= \text{la distance entre les deux masses}
\end{align*}

Ces deux équations sont à la base de la physique newtonienne en deux dimensions, or comme nous le savons, 
notre univers est en trois dimensions. Nous devons donc adapter ces équations pour qu'elles fonctionnent dans un espace 
tridimensionnel pour coller avec notre thème: la simulation de la réalité. 

Nous allons donc repartir de la loi de la gravitation universelle et y ajouter la notion de vecteurs pour obtenir une version de cette loi nous permettant de simuler en 3D.
\[
\hat{\textbf{r}} = \frac{\textbf{r}}{\norm{\textbf{r}}} = \frac{(x, y)}{r} \\
\]

puisque:

\[
r = \sqrt{x^2 + y^2}
\]
\[
\textbf{r} = (x, y)
\]

ceci nous permet d'obtenir les composantes de la force gravitationnelle dans chaque direction et ainsi d'obtenir la version vectorielle de la loi de la gravitation universelle:
\[
{F}_{grav} = -G \frac{M_{soleil} m_{bille}}{r^2} \hat{\textbf{r}} = -G \frac{M_{soleil} m_{bille}}{r^2} \frac{(x, y)}{r} = -G \frac{M_{soleil} m_{bille}}{r^3} (x, y)
\]

maintenant que nous avons notre formule de la force gravitationnelle vectorielle, nous pouvons l'intégrer dans la deuxième loi de Newton pour obtenir 
l'accélération de notre bille dans chaque direction:

\[
F_{grav} = ma = -G \frac{M_{soleil} m_{bille}}{r^3} (x, y)
\]

donc:

\[
a = -G \frac{M_{soleil}}{r^3} (x, y)
\]

\end{document}